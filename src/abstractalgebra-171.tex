Let $A$ and $B$ be two sets. The $\textit{Cartesian Product}$ is defined as the set of two-tuples contining every combination of elements from both sets. Formally, we express this as
$$\vskip 0.5em$$
$$A \times B = \{(a, b) \; \colon \; a \in A, \, b \in B\}$$
$$\vskip 0.6em$$
A $\textit{relation}$ between $A$ and $B$ is a subset $R \subseteq A \times B$. If $(a, b) \in R$, then "$a$ is related to $b$" and we can write $aRb$.
$$\vskip 0.6em$$
A function $\phi \, \colon \, X \rightarrow Y$ is a relation on $X \times Y$ where elements are written as $(x, y)$ and $y = \phi(x)$. In this form, $X$ is the $\textit{domain}$, and $Y$ is the $\textit{codomain}$. The $\textit{range}$ is defined as $\phi[X] := \{\phi(x) \, \colon \, x \in X\}$.
$$\vskip 0.6em$$
The number of elements $m$ in the set X represents the $\textit{cardinality}$ of $X$ and is represented by $|X|$.
$$\vskip 0.6em$$
To demonstrate two sets $X$ and $Y$ have the same cardinality, we must map each element of $X$ to each element of $Y$. Such pairings/mappings are called a $\textit{one-to-one\;correspondance}$ (also called $\textit{injective}$ in some texts).
$$\vskip 0.6em$$
A function $\phi \, \colon \, X \rightarrow Y$ is injective if $ \forall x_1, x_2 \in X, \, \phi(x_1) = \phi(x_2)$, then $x_1 = x_2$. A function is $\textit{onto}$ (also reffered to as $\textit{susrjective}$ in some texts) if the range of $\phi$ is $Y$. That is, $\phi[X] = Y$. A function which is both surjective and injective is called $\textit{bijective}$.
$$\vskip 0.6em$$
Bijective functions exhibit a property such that they have an inverse.
$$\vskip 0.6em$$
$X$ and $Y$ have the same cardinality when $\exists \phi \, \colon \, X \rightarrow Y$ such that $\phi$ is both surjective and injective.
$$\vskip 0.6em$$
An equivalence relation $R$ on a set $S$ is a relation (so $R \subseteq S \times S$), such that, $\forall x, y, z \in S$